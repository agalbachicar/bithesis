\subsubsection{Bet sizing}
\label{sec:methods_pipeline_bet_sizing}

Labels and metalabels will be used together with other features to train a secondary
model whose output estimated probability for each new label will be used to size
the bet for that event. One way to do this is to scale the predicted probability,
by the budget total and get value in currency for the bet. However, if we did so,
there would not be any awareness about other concurrent bets, so we might misuse
the available budget to run other concurrent bets. Lopez de Prado in chapter 10
of \cite{lopez_de_prado} proposes the following:

\begin{enumerate}
  \item Let $p_{(x)}$ be the probability of label $x$ that takes the one of the
        values in $[-1, 1]$.
  \item Run a statistical test where $H_{0}: p_{(x=1)} = 0.5$ with the statistic
        $z = \frac{p_{(x=1)} - 0.5}{\sqrt{0.5(1-0.5)}} = \frac{p_{(x=1)} - 0.5}{0.5}$
  \item Let the bet size $m$ be: $m = 2 \Phi_{(z)} - 1$ where $\Phi_{(z)}$ is
        the cumulative distribution function of the standard normal
        distribution. $m \in [-1; 1]$
\end{enumerate}

Once we have a vector of $m$ values, which has a one to one relationship with
each label $x$, we can average the bet with those concurrent triple barrier
windows as the come in the pipeline. Averaging does not change the bet size
for open windows, but changes the size for new bets that overlap with open
windows.

Finally, a portfolio manager might also consider bet size discretization by
means of setting an integer number of equally sized budget partitions. The
discretized value of the bet after the average is $m^* = round\Bigg( \frac{m}{d} \Bigg) d$
where $d = \frac{1}{Number of partitions}$. And the only remaining step when
running the strategy is to scale the budget by $m^*$ to have the final bet size
in currency. The goal of this step is to reduce jitter which causes overtrading.
Typical number of partitions are below ten.
