\subsubsection{Back testing}
\label{sec:methods_pipeline_backtesting}

This section focuses on explaining how this exercise would be 
carried out, the differences with respect to the implemented
back testing strategy and how benchmarking metrics are affected.
Also, metrics are explained.

\paragraph{Strategy} The strategy will start at certain date with an
initial budget of \$1,000,000 dollars (USD) to spend as both 
the primary and secondary model dictate. Every day, new features are gathered
and processed together with the historical data. Feed the selected set of
columns that conform the secondary model and the primary model determines
whether a new label and metalabel should be created and then the secondary model
proposes a probability for that metalabel. The bet sizing transformation will
scale and determine a dollar level for the bet. Depending on the position the
primary model proposes and the bet size, the following steps are conducted:

\begin{itemize}
  \item When placing a long position and the bet size is not zero:
  \begin{enumerate}
    \item Bet size dollars are taken from the portfolio at time $t$.
    \item Bet size dollars are used to buy bitcoins. The BTC/USD
          rate is the price at the moment where the order takes place. From the
          bet size dollars, the exchange buy fee is discounted from the bet
          size.
    \item We hold the long position until one of the three barriers
          of the triple barrier boundaries is crossed by the price 
          path. The primary model defined by hyperparameter 
          optimization a value for the stop loss, the profit taking and the
          holding period for the triple barrier, those values are computed based
          on the bitcoin price level at $t$.
    \item When the price series crosses any of the barriers, the 
          long position is closed and a sell fee is paid to the exchange.
    \item At this point, the value of the portfolio would be damaged 
          because of the buy and sell fees and benefited from the
          expected rise in bitcoin valuation measured in dollars. It
          is worth pointing out that the bet could end up having a
          net negative effect on the value of the portfolio.
  \end{enumerate}
  \item When placing a short position and the bet size is not zero:
  \begin{enumerate}
    \item Bet size dollars are taken from the portfolio at time $t$.
    \item A loan in bitcoins is requested. The bet size amount is
          used to pay the loan and pay the collateral amount.
    \item Those bitcoins are sold, consequently we get their 
          dollar valuation minus the sell fee of the exchange.
    \item We hold the short position until one of the three barriers
          of the triple barrier boundaries is crossed by the prices path. The
          primary model defined by hyperparameter optimization a value for the
          stop loss, the profit taking and the holding period for the triple
          barrier, those values are computed based on the bitcoin price level
          at $t$.
    \item When the price series crosses any of the barriers, the 
          short position is closed. In this case, bitcoins are bought
          to complement the collateral amount and return the loan interest.
    \item At this point, the value of the portfolio would be damaged 
          because of the buy and sell fees and the loan interest. However, it
          would benefit from the price drop and the short position.
  \end{enumerate}
\end{itemize}

The aforementioned procedures are iterated over the lifespan of the strategy and
must be monitored as any other strategy. Also, it is worth mentioning the risks
involved in the operation:

\begin{itemize}
  \item If the price path rapidly crosses the upper price level barrier before a 
        new price sample comes in (when evaluating sell orders) or it crosses the lower price
        level barrier before a new price sample comes in (when evaluating buy orders), there is
        an extra cost incurred due to a lack of live price monitor. It can be
        mitigated by having such monitor and an automatic order placing system.
  \item If having automatic control to place orders, the price path can simply
        not be enough to pay off the fees, even though the net return the price only is positive (accounting the transaction cost reduces the net profit from the trade).
\end{itemize}  

The portfolio manager will also have to consider how the collateral valuation
varies with time. Note that many bets time windows can overlap which
could cause budget starvation if highly confident bets co-occur.
Consequently, the collateral valuation will be affected by all short and
concurrent positions and with the price variations of the asset under management
and will reduce the effective return of each bet because of the compromised
savings to pay back the loan. However, it is worth mentioning that in case of a
an undesired rise in prices, the collateral will help to reduce the impact.

In this research project, we make the following assumptions:

\begin{itemize}
  \item Buy and sell exchange fees might vary over time. For simplicity, they
        are kept the same for the entire back testing period.
  \item Loan interests vary their every day. For
        simplicity, they are kept the same for the entire back testing period.
  \item Collateral budgets are not considered. Instead, a penalty to the loan
        interest is added.
  \item Because of considering the collateral budget as an extra interest, the
        value of the portfolio does not benefit from intra bet windows
        bitcoin price variation.
  \item Orders might not get processed during high price corrections. Those
        scenarios are not contemplated either.
\end{itemize}

In general, all of these simplifications will imply higher performance which is
perceived as excess returns.

As a reference, table \ref{table:benchmark_fees} contains the fees and rates involved in the strategy.

\begin{table}[H]
  \centering
  \begin{tabular}{| c | c |} 
    \hline
    \multicolumn{2}{|c|}{Fees} \\
    \hline
    Fee & Value \\
    \hline
    Buy & 0.10\% \\
    \hline
    Sell & 0.10\% \\
    \hline
    Loan interest & 5\% \\
    \hline
  \end{tabular}
  \caption{Financial metrics benchmark.}
  \label{table:benchmark_fees}
\end{table}

Buy and sell fees were take from Binance \cite{binance_fees} and loans from 
DefiRate \cite{defi_rate_loans} where an average has been done and a 1\% extra
has been added.

\paragraph{Benchmark} A base strategy is required to provide a reference and
compare against the output strategy of this research. Buying bitcoins the first
day and holding them until the last day of the testing period will be used. This
is an obvious choice because machine learning enhanced momentum is used over this
asset and would set a valid benchmark for this strategy.

\paragraph{Metrics} The secondary model will be measured with two metrics which
are appropriate for classification problems. The F1-score as a measure of the 
accuracy of the classifier. Then, to measure classifier performance, negative
logarithmic loss will be used to assess how well predicted probabilities adjust
to validation labels.

The strategy will be evaluated by the following metrics which also cover different
types:

\begin{itemize}
  \item Performance metrics:
  \begin{itemize}
    \item Win loss ratio: the ratio between positive returns and negative returns.
          The bigger the ratio, the better.
    \item Win rate: the ratio of positive returns over the total number of
          returns. The bigger the ration, the better.
    \item Average return: the arithmetic average of all returns. It is an estimate of the
          true mean return.
  \end{itemize}
  \item Efficiency metrics:
  \begin{itemize}
    \item Sharpe Ratio: measures the ratio between the average sample returns minus a risk free interest rate\footnotemark, and
          the sample standard deviation. It provides an estimate of the true
          Sharpe Ratio.
    \item Probabilistic Sharpe Ratio: expresses the probability that estimated
          Sharpe Ratio (see above) is higher than a benchmark Sharpe Ratio. See
          below for an in detail discussion.
    \item Sortino ratio: it is similar to the Sharpe Ratio in the sense that provides a
          a risk adjusted return ratio but it penalizes negative returns. It is
          generally a better choice to compare skewed return distributions. As the
          Sortino ratio increases the strategy becomes more appealing for a given
          benchmark target return. See \cite{sortino_a_sharper_ratio} for an
          analysis and explanation of the ratio.
  \end{itemize}
  \item Implementation shortfall metrics:
  \begin{itemize}
    \item Return on execution costs: ratio between the sum of all net returns
          (after operation costs) and the sum of all transaction costs. The bigger
          the ratio is, the more confident a portfolio manager will be about the
          strategy. Transaction costs cannot be known beforehand and high values
          cover for increased costs at execution time.
  \end{itemize}
  \item General metrics:
  \begin{itemize}
    \item Correlation to underlying: the correlation between the strategy and the
          underlying investment universe. As it gets closer to one or minus one,
          the closer it is to the benchmark strategy.
    \item Volatility: the standard deviation of the returns. It is an estimate of
          the true return standard deviation.
  \end{itemize}
\end{itemize} 

\footnotetext{Using US treasury yield annual curves \cite{treasury_yield}, one can observe that risk free interest rates vary from 0.05\% to 2.74\%. It means that the daily risk free interest rate varies from $2.73 x 10^{-6}$ to $7.4 x 10^{-5}$. It was simply considered zero.}

Moreover, it is important to discuss the Probabilistic Sharpe Ratio. It was
introduced in \cite{sharpe_paper} which compliments the information of the
estimated Sharpe Ratio which is usually provided in financial benchmarks.
For both strategies, the benchmark and the strategy under test, the true mean
and standard deviation of the Sharpe Ratio is and will remain unknown. Back
testing over past or generated data will only provide an estimate that adjusts
to the sample consequently, the Probabilistic Sharpe Ratio helps to understand
which is the probability that the estimated Sharpe Ratio is bigger than a set of
target ratios.

Furthermore, the estimate of the Sharpe Ratio mean comes from non IID return samples. Hence, Bailey and 
Lopez de Prado introduced an adjustment by skewness and kurtosis to the standard
error used in the statistical test. Then, it is appropriate to use a series of
target Sharpe Ratio to compare with and determine the probability that the
estimated Sharpe Ratio of our strategy is bigger for proposed target ratios. As a
reference, the Sharpe Ratio and the Probabilistic Sharpe Ratio formulas are copied
below.

\[\hat{SR} = \frac{\hat{r}}{s_{r}}\]

Where:

\begin{itemize}
  \item $\hat{SR}$ is the estimated Sharpe Ratio.
  \item $\hat{r}$ is the average return.
  \item $\hat{s_{r}}$ is the sample standard deviation of the returns.
\end{itemize}

\begin{equation}
  \label{eqn:prob_sharpe_ratio}
  \hat{PSR}_{(SR^*)} = \Phi \Bigg( \frac{(\hat{SR} - SR^*) \sqrt{N-1}}{\sqrt{1 - \hat{\gamma_3}\hat{SR} + \frac{\hat{\gamma_4} - 1}{4}\hat{SR}^2}} \Bigg)
\end{equation}

Where:

\begin{itemize}
  \item $\hat{PSR}_{(SR^*)}$ is the Probabilistic Sharpe Ratio for the reference
        Sharpe Ratio $SR^*$.
  \item $N$ is the number of returns in the sample to compute the index.
  \item $\hat{\gamma_3}$ is the sample return skewness.
  \item $\hat{\gamma_4}$ is the sample return kurtosis.
  \item $\Phi()$ is the cumulative distribution function of the standard Normal
        distribution.
\end{itemize}

\paragraph{Taxation} A potentially relevant “transaction cost” relates to taxes.
For example, in the US, and depending on the tax bracket, federal taxes for
short-term capital gains are taxed at up to 37\%\footnotemark . Capital gain taxes would
disfavor active trading on a highly volatile asset in our results.

\footnotetext{See \href{https://www.irs.gov/pub/irs-drop/rp-19-44.pdf}{26 CFR 601.602: Tax forms and instructions.}}