\subsection{Problem description}
\label{sec:intro_problem_description}

In the last decades, the notorious technology improvements enabled several industries to grow exponentially. Specifically, in the finance industry we can identify a more recent trend that involves considerable investment efforts in the quantitative and algorithmic trading (see \cite{blackrock_investment} to learn more about the BlackRock case). It is interesting for this research to focus initially on two types of solutions from the above group: high frequency trading (see \cite{hft_intro}), known as HFT, and machine learning for trading \cite{machine_learning_in_finance}. The first group exploits the computational power to arbitrate between securities whereas the other exploits the vast amount of available data through data mining techniques and statistical models.

Different technological improvements enabled the spread of these techniques in the financial industry. First, having more and more data to process required more and more storage which started to cost less and less per storage unit, i.e. \$/GB (\cite{mkomo_cost_per_gb_updated} and the updated version \cite{mkomo_cost_per_gb_updated}). Secondly, the widespread use of mobile devices supported by a more and more connected world (\cite{itu_inet_access}) enabled faster data transfers reducing distance, information lead time, or just broadly speaking the connectivity costs. Another vertical that supported this growth was the raise in computational power which followed Moore's Law \cite{moore_law} allowed more and more complex algorithms to be run in acceptable time (with respect to the needs at hand) enabling old theoretical solutions to have a practical one. For example, Roosenblatt's perceptron dates from 1958 \cite{rosenblatt_perceptron} but it was not after some decades that commercial applications of  those neural networks were made available.

Moreover, the digitalization process and internet access growth \cite{owidinternet} allow more people to participate in markets, access information and make new decisions. The behavior of each economic agent, i.e. individual, in the market became of tremendous importance. It is not anymore a handful of people making decisions but entire societies. Behavioral economic analysis is required to better understand market fluctuations and its evolution. For example, herds and bubbles have been registered for centuries \cite{mackay} to nowadays with two of the most recent and bigger bubbles: the Dotcom bubble \cite{dot_com_bubble} and the US financial crisis in 2008 \cite{financial_crisis_2008}.

In parallel, another phenomenon occurred in 2008 when Satoshi Nakamoto released the Bitcoin paper \cite{bitcoin} and then in 2009 when the Bitcoin software release was published \cite{bitcoin_release}. Pushed by Bitcoin, a new technology ecosystem appeared in the last decade backed up by the blockchain technology. As of February 2021, it is said that over 4,000 cryptocurrencies are available but only twenty of them concentrate the 90\% of the market \cite{statista_crypto}. A document of the World Bank \cite{world_bank_bitcoin} showed that Bitcoin transfers were used for gambling and dark web transactions but it gained attraction in 2016 to ease the process international transactions and then to finance private endeavors. Governments started to experiment with blockchain technologies and what started a decade ago as a niche experiment, it reached a market capitalization of \$1,022,439,972,862 in March 9\textsuperscript{th} 2021 \cite{coinmarket_market_cap_bitcoin}.

Bitcoin and other cryptocurrencies are part of a broader collection named cryptoassets \cite{cryptoassets_book}. The term currency is subject to discussion but omitting the subjective appreciations in the academy in favor or against, we should also consider cryptocommodities and cryptotokens. In \cite{cryptoassets_book} there is a good introduction about these terms and thorough examples for each of them based on their used and recent attraction.

One of the main characteristics about Bitcoin is that its total emission and its emission rate has been defined by design. Every four years the incentive to miners is reduced to a half (it started in 50BTC). This event is called halvening and introduces a structural change in the market because incentives suffer dramatic changes (they are cut down to a half). As it will be later explained, these events involve a high volatility in Bitcoin price as well as a change in market's regime. Strategies around this type of asset should either stay away of halvenings or consider them somehow to avoid important losses. In \cite{lopez_de_prado}, structural breaks are considered to inform models with statistical tests about structural breaks in market that would yield to the "best risk / reward ratios".

This research focuses on a building a strategy development pipeline to build, train and evaluate financial trading strategies and will be exercised with Bitcoin. A primary model based on momentum will provide the main trading signals and a secondary machine learning model will provide the bet size. Financial indexes (such as price, volume and volatility), structural break indexes (such as Supremum Augmented Dickey-Fuller), Bitcoin related indexes (such as stock to flow and number of new addresses ) and social animosity indexes (such as fear and greed index) will be evaluated to improve the secondary model performance. Finally, from backtesting procedures metrics will be determined to assess the strategy performance.

This document has the following outline:
\begin{itemize}
    \item Section \ref{sec:intro} presents the problem to solve, provides context about each involved discipline, comments about the state of the art and defines the scope of this research.
    \item Section \ref{sec:materials} presents the used data sources.
    \item Section \ref{sec:methods} analyzes in detail the features and describes the pipeline and successive iterations over it.
    \item Section \ref{sec:results} presents the results obtained. Pure machine learning model results are separated from financial results.
    \item Section \ref{sec:conclusion} discusses the results and wraps the document.
    \item Section \ref{sec:future_work} presents some unresolved questions and potential lines of work to continue with this research effort.
\end{itemize}